\cleardoublepage
\chapter{Conclusiones y trabajos futuros}
\label{chap:conclusiones-trabajos-futuros}

% Reflexión sobre el trabajo realizado, qué objetivos se han cumplido y qué aspectos quedan pendientes para una futura ampliación del proyecto. Además, se deben incluir unas conclusiones personales indicando lo que ha supuesto para el alumno la realización del trabajo. Entre 2 y 4 páginas.

La realización de este trabajo me ha permitido conocer la complejidad y los beneficios de realizar un proyecto siguiendo buenas prácticas desde la definición a la implementación pasando por la etapa de diseño y el control del progreso.

Surgío de las dificultades que sufren equipos multidisciplinares digitales para comunicarse y entender las decisiones de cada capa. Parte fundamental para el trabajo de un desarrollador a parte de conocer, entender e implementar correctamente las tecnologias actuales y las buenas prácticas de generación de código. Es una habilidad que en cualquier empresa del sector o en cualquier proyecto que se realiza en grupo hay que cuidar y mejorar. (aplicar)

por supuesto aprender a implementar una aplicación nativa y el uso de tecnologías actuales.

%%% internalizacion, el uso de IA para recomendación y la prediccion de articulos que necesita el usuario, incluir test de E2E y de cobertura


