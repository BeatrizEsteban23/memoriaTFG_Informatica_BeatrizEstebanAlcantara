\cleardoublepage
\chapter{Conclusiones y trabajos futuros}
\label{chap:conclusiones-trabajos-futuros}

% Reflexión sobre el trabajo realizado, qué objetivos se han cumplido y qué aspectos quedan pendientes para una futura ampliación del proyecto. Además, se deben incluir unas conclusiones personales indicando lo que ha supuesto para el alumno la realización del trabajo. Entre 2 y 4 páginas.

\section{Objetivos cumplidos}
\label{sec:objetivos-cumplidos}
En esta sección voy a diferenciar dos objetivos claros que se han presentado a lo largo de este proyecto.

En primer lugar, comprender y analizar las funciones y responsabilidades de cada rol en un equipo digital. Se ha cumplido el objetivo de explorar y aplicar metodologías y herramientas conmunmente utilizadas por el equipo de diseño y de producto, de las que desde el lado técnico solemos tener menos compresión pero que impactan directamente en nuestro trabajo. Y por supuesto, refrescar y aplicar todo el conocimiento adquirido durante la realización del grado que aplica a estos dos roles. Esto ha permitido dar una solución centrada en el usuario para facilitar una tarea tan cotidiana y personalizable como la gestión de la compra en el hogar.

Y en segundo lugar, la exploración de todas las tecnologías y buenas prácticas que actualmente aplican al desarrollo de aplicaciones nativas, especialmente de Android. Al entenderlas, otro de los objetivos cumplidos ha sido saber conocer qué opciones hay y cómo aplicar las que realmente eran necesarias o aportaban un valor en el desarrollo de Pinche.

Tanto en el prototipo funcional en figma como en la planificación de historias de usuario que se plantea en la etapa inicial de este proyecto, hay una discrepancia entre lo planteado y lo que se ha llegado a implementar en la aplicación.

Cumpliendo con toda la funcionalidad prevista para las secciones de gestión de la cuenta del usuario en la aplicación y de la lista de la compra con sus respectivos articulos. Pero sin llegar a implementar: TAL TAL TAL. Esto se debe a una estimación que no preveía el tiempo y esfuerzo requerido para adquirir y aplicar conocimientos nuevos en un campo tan amplio como el desarrollo nativo de aplicaciones.

\section{Trabajos futuros}
\label{sec:trabajos-futuros}

Tras la implementación de una base sólida de la aplicación Pinche, las lineas de trabajo futuro que considero que tendría sentido implementar son:

\begin{itemize}
    \item \textbf{Funcionalidades planteadas pero no implementadas:} Terminar de implementar la gestión completa de recetas e invitados.
    \item \textbf{Internalización:} Añadir la opción de cambiar de idioma dentro de la aplicación, ampliando así el mercado de posibles usuarios.
    \item \textbf{Incorporación de fotografías:} Ofrecer la opción de añadir fotografías de los artículos, las recetas y de los invitados.
    \item \textbf{Incorporar funcionalidades gracias a la Inteligencia Artificial (IA):} Explorar cómo la IA puede impactar en nuevas funcionalidades en la aplicación. Por ejemplo, recomendar alimentos o recetas al usuario o predecir el gasto de un alimento para añadirlo de nuevo a la lista de la compra.
    \item \textbf{Accesibilidad:} Incluir todo el código necesario para que la aplicación sea accesible para todos los usuarios: etiquetas, cambio de diseño, etc.
    \item \textbf{Ampliar la cobertura de tests:} Terminar por cubrir por completo la aplicación con los tests unitarios y de interfaz ya planteados y añadir tests de extremos a extremo (E2E). Los tests E2E comprueban flujos completos: desde la interacción del usuario, la navegación, el estado de la interfaz y la manipulación y la persistencia de los datos. Incorporar también tests de cobertura que permitan conocer qué porcentaje de código se testea en las pruebas.
\end{itemize}

\section{Conclusiones personales}
\label{sec:conclusiones-personales}

La realización de este trabajo me ha permitido conocer la complejidad y los beneficios de realizar un proyecto siguiendo buenas prácticas desde la definición a la implementación pasando por la etapa de diseño y el control del progreso.

He entendido las dificultades que sufren equipos multidisciplinares digitales para comunicarse y la importancia de entender las decisiones de cada rol. Creo que es una parte fundamental y que facilita el trabajo de un desarrollador a parte de conocer, entender e implementar correctamente las tecnologias actuales y las buenas prácticas de generación de código. En mi opinión, la comunicación entre roles es una habilidad que en cualquier empresa del sector o en cualquier proyecto que se realiza en equipo hay que aplicar, cuidar y mejorar.

Ha sido mi primer contacto con la implementación de una aplicación nativa en Android. Puede que aprender el lenguaje de programación Kotlin y el comprender y aplicar buenas prácticas como: la arquitectura MVVM, la inyección de dependencias, la navegación, el estado de los componentes de Jetpack Compose o realización de test unitarios y de UI ha sido la parte a la que más tiempo he dedicado y que más esfuerzo me ha requerido para no perderla de vista. Sin embargo, como siempre pasa con las buenas prácticas, me he beneficiado de aplicar todas ellas durante todo el desarrollo de la aplicación.

He aprendido muchos conceptos que impactan por completo en mi carrera profesional, añadiendo a mis conocimientos de desarrollo Front-End la parte nativa de aplicaciones, que tiene una alta demanda en el sector, abriendome así las puertas a nuevas oportunidades de empleo y de desarrollo en mi carrera profesional.