\cleardoublepage % empezamos en página impar
\chapter{Objetivos} % título del capítulo (se muestra)
\label{chap:objetivos} % identificador del capítulo (no se muestra, es para poder referenciarlo)

El objetivo principal de este proyecto es comprender y aplicar todo el proceso que conlleva la creación de una aplicación desde cero. En este caso, nos centraremos en cómo facilitar al usuario la tarea de realizar la compra doméstica, tanto de alimentos como de productos del hogar, de manera eficiente y organizada. También se tendrá en cuenta cómo gestiona las recetas y los invitados que van a comer a su hogar.

Para ello el trabajo se centrará en el desarrollo de una aplicación móvil nativa para el sistema operativo Android, denominada \textit{Pinche}. Pinche tiene cómo objetivo facilitar la tarea expuesta previamente a los usuarios.

La forma de alcanzar este objetivo consistirá en recorrer cada una de las etapas del proceso de desarrollo, adoptando los distintos roles que forman parte de un equipo digital: experiencia de usuario (UX), producto y desarrollo. Aunque el núcleo del trabajo se centrará en la parte técnica de desarrollo, también se tendrán en cuenta aspectos clave de diseño y definición del producto.

Adoptar esta perspectiva integral permite no solo aprender a programar, sino también a tomar decisiones informadas sobre las necesidades reales de los usuarios, los problemas que enfrentan en su día a día y cómo una solución tecnológica puede aportar valor real.

La motivación del proyecto radica también  en aplicar y reforzar conocimientos adquiridos durante el grado, desde el diseño de interfaces hasta la integración de servicios en la nube. Pero sin permanecer como técnicos o programadores aislados que no tienen el contexto del proyecto que llevan a cabo y que entienden el por qué de las decisiones que se toman en su conjunto.

Desde ese punto de vista, el puramente técnico, se ha optado por emplear tecnologías modernas del ecosistema Android: el lenguaje de programación Kotlin, el framework Jetpack Compose para la construcción de interfaces declarativas y los servicios de Firebase —específicamente Firestore como base de datos y Firebase Authentication para la gestión de usuarios—. Estas tecnologías permiten un desarrollo flexible y están alineadas con las demandas actuales del sector.

Desde la perspectiva de diseño UX se ha adoptado una combinación de la metodología Design Thinking para la definición inicial del problema y el uso de la herramienta Figma para llevar acabo un diseño final intuitivo y que sea fácilmente iterable.

Por último, desde el lado de producto y gestión del proyecto, se aplica la metodología Lean Startup para validar hipótesis y metodologías ágiles como Scrum para organizar el desarrollo en iteraciones.

El presente trabajo ofrece una oportunidad para adquirir experiencia práctica en un entorno controlado, simulando las condiciones reales del desarrollo de software en equipos multidisciplinares. Asumir diferentes roles permite entender los retos de la comunicación entre perfiles técnicos y no técnicos, la importancia de la empatía hacia el usuario final, y el valor de trabajar con un enfoque centrado en el producto y el usuario.

En resumen, el proyecto se plantea como una oportunidad para consolidar conocimientos técnicos, mejorar habilidades de diseño y comunicación, y resolver una necesidad real con impacto práctico. A través del desarrollo de esta aplicación, se busca no solo alcanzar los objetivos académicos del Trabajo de Fin de Grado, sino también generar una solución útil que podría ser utilizada por muchas personas en su vida diaria.
