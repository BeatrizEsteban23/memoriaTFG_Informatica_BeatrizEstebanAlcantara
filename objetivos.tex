\cleardoublepage % empezamos en página impar
\chapter{Objetivos} % título del capítulo (se muestra)
\label{chap:objetivos} % identificador del capítulo (no se muestra, es para poder referenciarlo)

El objetivo principal de este proyecto es comprender y aplicar todo el proceso que conlleva la creación de una aplicación desde cero. En este caso, nos centraremos en cómo facilitar al usuario la tarea de realizar la compra doméstica, tanto de alimentos como de productos del hogar, de manera eficiente y organizada.

Este trabajo se basa en el desarrollo de una aplicación móvil nativa para el sistema operativo Android, denominada \textit{Pinche}, que busca ayudar a los usuarios a planificar su compra semanal, registrar los productos que necesitan y gestionar su consumo.

La forma de alcanzar este objetivo consistirá en recorrer cada una de las etapas del proceso de desarrollo, adoptando los distintos roles que forman parte de un equipo digital: experiencia de usuario (UX), producto y desarrollo. Aunque el núcleo del trabajo se centrará en la parte técnica, también se tendrán en cuenta aspectos clave de diseño y definición del producto.

Adoptar esta perspectiva integral permite no solo aprender a programar, sino también a tomar decisiones informadas sobre las necesidades reales de los usuarios, los problemas que enfrentan en su día a día y cómo una solución tecnológica puede aportar valor real.

La motivación del proyecto radica precisamente en cubrir este vacío mediante el desarrollo de una aplicación intuitiva, personalizada y eficiente, centrada en mejorar la experiencia del usuario. Además, permite aplicar y reforzar conocimientos adquiridos durante el grado, desde el diseño de interfaces hasta la integración de servicios en la nube.

Desde el punto de vista técnico, se ha optado por emplear tecnologías modernas del ecosistema Android: el lenguaje de programación Kotlin, el framework Jetpack Compose para la construcción de interfaces declarativas y los servicios de Firebase —específicamente Firestore como base de datos y Firebase Authentication para la gestión de usuarios—. Estas tecnologías permiten un desarrollo flexible y están alineadas con las demandas actuales del sector.

En cuanto a la metodología, se ha adoptado una combinación de Design Thinking para la definición inicial del problema, Lean Startup para validar hipótesis de producto, y metodologías ágiles como Scrum para organizar el desarrollo en iteraciones.

El presente trabajo ofrece una oportunidad para adquirir experiencia práctica en un entorno controlado, simulando las condiciones reales del desarrollo de software en equipos multidisciplinares. Asumir diferentes roles permite entender los retos de la comunicación entre perfiles técnicos y no técnicos, la importancia de la empatía hacia el usuario final, y el valor de trabajar con un enfoque centrado en el producto.

En resumen, el proyecto se plantea como una oportunidad para consolidar conocimientos técnicos, mejorar habilidades de diseño y comunicación, y resolver una necesidad real con impacto práctico. A través del desarrollo de esta aplicación, se busca no solo alcanzar los objetivos académicos del Trabajo de Fin de Grado, sino también generar una solución útil que podría ser utilizada por muchas personas en su vida diaria.
