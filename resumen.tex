\chapter*{Resumen}
%\addcontentsline{toc}{chapter}{Resumen} % si queremos que aparezca en el índice
\markboth{RESUMEN}{RESUMEN} % encabezado

Este Trabajo de Fin de Grado presenta el desarrollo de \textbf{Pinche}, una aplicación móvil nativa Android cuyo objetivo es facilitar la planificación y organización de la compra doméstica. Se ofrece una visión completa del proceso de desarrollo desde la perspectiva de distintos roles: UX (Experiencia de usuario), producto y técnico, centrándonos especialemente en este último. El objetivo de entender este proceso es poner en relieve la importancia de que cada pieza que conforma un equipo digital entienda los requisitos del producto y el razonamiento que hay detrás de cada uno de ellos a la hora de implementarlo.

El proyecto se ha desarrollado siguiendo una arquitectura basada en el patrón MVVM, utilizando tecnologías modernas como Kotlin (lenguaje nativo de Android), Jetpack Compose, Hilt y Firebase (Firestore y Authentication) y aplicando buenas práticas de diseño. También se ha seguido una estrategia de testeo que incluye pruebas unitarias y de interfaz. El diseño de la interfaz se ha prototipado en Figma, y la gestión del proyecto se ha organizado en Trello.

El trabajo se enmarca en el contexto del desarrollo de software actual y el estilo de vida social. Poniendo de manifiesto cómo las aplicaciones móviles pueden impactar en la mejora de la calidad de vida de las personas, optimizando tareas de su día a día, y la importancia de colocar al usuario en el centro del proceso de su desarrollo.
