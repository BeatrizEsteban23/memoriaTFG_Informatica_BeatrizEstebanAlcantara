\chapter*{Resumen}
%\addcontentsline{toc}{chapter}{Resumen} % si queremos que aparezca en el índice
\markboth{RESUMEN}{RESUMEN} % encabezado

Aquí viene un resumen del proyecto. Ha de constar de tres o cuatro párrafos, donde se presente de manera clara y concisa de qué va el proyecto.
Han de quedar respondidas las siguientes preguntas:

\begin{itemize}
  \item ¿De qué va este proyecto? ¿Cuál es su objetivo principal?
  \item ¿Cómo se ha realizado? ¿Qué tecnologías están involucradas?
  \item ¿En qué contexto se ha realizado el proyecto? ¿Es un proyecto dentro de un marco general?
\end{itemize}

Lo mejor es escribir el resumen al final.

Este Trabajo de Fin de Grado presenta el desarrollo de \textbf{Pinche}, una aplicación móvil nativa para Android cuyo objetivo es facilitar la planificación y organización de la compra doméstica. A través de sus tres módulos principales —listas de la compra, recetas e invitados—, la aplicación permite gestionar productos, calcular ingredientes en función del número de comensales y registrar preferencias alimentarias, proporcionando una solución práctica para mejorar la experiencia del usuario en el hogar.

El proyecto se ha desarrollado siguiendo una arquitectura basada en el patrón MVVM, utilizando tecnologías modernas como Kotlin, Jetpack Compose, Hilt y Firebase (Firestore y Authentication). Se han aplicado buenas prácticas de diseño y una estrategia de testeo completa que incluye pruebas unitarias, de interfaz, integración, E2E y cobertura de código. El diseño de la interfaz se ha prototipado en Figma, y la gestión del proyecto se ha organizado con herramientas ágiles como Trello.

El trabajo se enmarca en el contexto del desarrollo de software actual, en el que las aplicaciones móviles tienen un rol esencial en la mejora de la calidad de vida de las personas. Se ha realizado un análisis del mercado de aplicaciones, se ha identificado la oportunidad de mejora respecto a otras soluciones existentes como Bring! o Listonic, y se ha desarrollado una solución adaptada a las necesidades reales de usuarios particulares.
